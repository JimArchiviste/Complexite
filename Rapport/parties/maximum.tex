\section{Génération d'un sous-graphe maximum}

Regroupés ici sont les deux algorithmes permettant d'obtenir des sous-graphes maximum. Le premier utilise une méthode exacte mais particulièrement lourde tandis que le second se contente d'une heuristique arbitraire offrant un sous-graphe s'approchant d'un maximum mais sans garantie précise sur sa `qualité'.

\subsection{Méthode exacte}

\subsubsection{Principe}

\begin{verbatim}
void maximum_exact(graphe G, sous-graphe S){
    sous-graphe tmp
    ensemble-de-sous-graphes ens
    maximum_exact_rec(G,tmp,&ens,0)
    extract_maximum(ens,S)
}

void maximum_exact_rec(graphe G, sous-graphe S, ensemble-de-sous-graphes ens, entier prof){
    booléen continuer = vrai
    entier n = taille de G
    si(S non-vide)
        si(is_desert(S,G))
            si(is_maximal(S,G))
                memorize(S,ens)
                continuer = faux
            fsi
        sinon
            continuer = faux
        fsi
    fsi
    si(continuer)
        si(prof<n)
            sous-graphe S1,S2
            copy(S,S1)
            copy(S,S2)
            S2[prof] = 1
            maximum_exact_rec(G,S1,ens,prof+1)
            maximum_exact_rec(G,S1,ens,prof+1)
		    fsi
    fsi
}
\end{verbatim}

Voici une bréve explication des fonctions de plus bas niveau qui sont appelées par celles détaillées ici : 
\begin{itemize}
	\item{\verb|extract_maximum|} prend en paramètre une liste chainée de sous-graphes et un sous-graphe vide ; elle copie un des sous-graphes de taille maximum de l'ensemble dans le sous-graphe vide
	\item{\verb|memorize|} enregistre un sous-graphe donné dans une liste chainée de sous-graphes en créant un nouveau maillon et l'y copiant
	\item{\verb|copy|} copie, comme son nom l'indique, un sous-graphe source dans un sous-graphe destination, écrasant tout ce qui se trouvait dans ce dernier auparavant
\end{itemize}
Le principe de la partie récursive de l'algorithme est de vérifier dans un premier temps si on est dans un cas terminal, c'est-à-dire si le sous-graphe passé en paramètre est désert maximal. Si c'est le cas, on le stocke dans notre liste de sous-graphes et il n'y a pas d'appel récursif.

Si ce n'est pas le cas (mais si le sous-graphe est tout de même désert, sinon il est inutile de continuer à développer cette chaîne d'appels) on crée deux branches, une où l'un des sommets de G est ajouté à S en fonction de la profondeur à laquelle on se trouve dans la suite d'appels récursifs, et une où S n'est pas modifié. 

Puis deux appels récursifs sont faits, avec une profondeur incrémentée et chacun des deux sous-graphes mentionnés précédemment. Incidemment, nous allons créer tous les sous-graphes possibles de G et tester chacun d'entre eux pour mettre de côté tous les maximaux.

Une fois ces appels récursifs terminés, c'est la fonction \verb|extract_maximum| qui parcourt tous les résultats obtenus et en extrait un de taille la plus grande. 

\subsubsection{Validité}

Tout d'abord, notre fonction récursive atteindra toujours son cas terminal. La mécanique est la même que pour la fonction de génération d'un sous-graphe maximal : nous rajoutons des sommets au sous-graphe, si bien que l'on arrive nécessairement dans un des deux cas terminaux  possibles (soit le sous-graphe cesse d'être désert, soit il devient désert maximal). 

La principale différence vient du fait que nous testons exhaustivement les sous-graphes possibles plutôt que de suivre un seul chemin  pseudo-aléatoire. 

Le bon fonctionnement de la fonction \verb|maximum_exact| découle de la supposée validité des fonctions \verb|is_maximal| et \verb|is_desert| ; en effet, ce sont celles-ci qui permettent de stopper la récursion et de reconnaître les sous-graphes à mémoriser. 

\subsubsection{Complexité}

La complexité de cette fonction est plus longue à calculer que les précédentes. Chaque appel récursif en engendrant 2 nouveaux jusqu'à une profondeur maximum de n, où n est la taille de G, nous avons au final $2^n$ appels récursifs. 

n d'entre eux sont des cas terminaux donc la complexité est de l'ordre de celle de la fonction \verb|memorize| (à savoir $\theta(NMAX)$ où NMAX est le nombre maximum de sommets d'un graphe).

$2^n-n$ sont des cas non terminaux d'une complexité de l'ordre de celle de \verb|copy| (à savoir $\theta(NMAX)$).

Ainsi l'ensemble de l'arbre d'appels récursifs est d'une complexité globale de l'ordre de $\theta(2^n \times NMAX)$. 

Outre  ces appels récursifs, la fonction \verb|maximum_exact| fait appel à \verb|extract_maximum| qui appelle autant de fois \verb|copy| qu'il y a de sous-graphes maximaux dans G, mais je n'ai pas su calculer la complexité de cette fonction.

Pour conclure, je pense donc que la fonction \verb|maximum_exact| est d'une complexité exponentielle par rapport à la taille du graphe G en entrée, mais sans en être sûr :

$\theta(2^n)$

\emph{Temps d'exécution :} 0.001335s

\subsection{Méthode incomplète}

\subsubsection{Principe}

\begin{verbatim}
void maximum_partial(graph G, sous-graphe S){
    entier index
    sous-graphe sommets_intedits
    tant que(!is_maximal(G,S))
        index = vertex_with_less_edges(G,sommets_interdits)
        sommets_interdits[index] = 1
        S[index] = 1
        si(!is_desert(G,S)) S[index] = 0
    ftantque
}
\end{verbatim}
\verb|vertex_with_less_edges| est une fonction qui, pour un graphe G et un sous-graphe de sommets à ne pas prendre en compte I, va retourner l'indice du sommet de G privé de I ayant le plus petit nombre d'arcs incidents. 

L'idée est que, pour avoir de bonnes chances de générer un sous-graphe maximum, nous allons regarder le sommet avec le moins d'arcs incidents et l'ajouter à notre sous-graphe. 

Si le sous-graphe est toujours désert, nous poursuivons les tours de boucle jusqu'à ce qu'il soit maximal,  mais il faut non seulement rajouter ce sommet à l'ensemble des sommets à ne plus traiter, mais aussi tous ses sommets voisins. 

Si le sous-graphe n'est plus désert, le sommet en est retiré pour revenir à un sous-graphe désert, mais il est quand même rajouté à la liste des sommets interdits pour ne pas être traité une seconde fois par la suite (sans quoi l'algorithme tournerait en boucle à traiter encore et encore le même sommet). 

Quand on a atteint un sous-graphe désert maximal comportant les sommets avec le moins d'arcs incidents possibles, on le retourne comme étant (probablement) maximum. 

\subsubsection{Validité}

En supposant la validité de \verb|vertex_with_less_edges|, au pire des cas tous les sommets de G seront traités et ajoutés à S, mais nous finirons nécessairement par avoir un sous-graphe désert maximal. 

Or, ne cherchant pas l'exhaustivité ici, nous conservons le premier désert maximal rencontré comme résultat final. 

\subsubsection{Complexité}

Dans \verb|maximum_partial|, au maximum n tours de boucle sont effectués (dans le cas où tous les sommets doivent être parcourus pour obtenir un sous-graphe désert maximal). Chaque tour de boucle comporte 2 appels, \verb|vertex_with_less_edges| et \verb|is_desert|.

On a déjà vu que la seconde fonction a une complexité en $\theta(n^2)$. Le première a également une complexité en $\theta(n^2)$ car elle comporte 2 boucles imbriquées (l'une étant de  la forme pour i de 0 à n, l'autre pour j de i+1 à n), et ces boucles effectuent un traitement en temps constant. 

Notre fonction \verb|maximum_partial| a donc une complexité globale en :

$\theta(n^3)$

\emph{Temps d'exécution :} 0.001364s

