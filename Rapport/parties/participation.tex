\section{Participations au projet \& configuration de tests}

Le tableau \textsc{tab \ref{contributions}} a été difficile à établir car la participation de chacun n'est pas toujours précisément mesurable, notamment quand il s'agit de réfléchir à plusieurs sur l'écriture d'un algorithme ; cela a notamment été le cas pour la fonction maximum incomplète qui, bien que probablement triviale, nous a demandé un certain temps de réflexion.  
\begin{table}
			\begin{tabular}{|l|l|l|l|l|}
				\hline
				\textbf{Algorithme} & \textbf{Auteur du code} & \textbf{Auteur du rapport} & \textbf{Complexité} & \textbf{Validité}\\ \hline
				is\_desert & Guélaud & Alexandre & Guélaud & Guélaud \\
				is\_maximal & Ghislain & Alexandre & Ghislain & Ghislain \\
				maximal & Loïc & Alexandre & Loïc & Loïc \\
				maximum\_exact & Alexandre & Alexandre & Alexandre & Alexandre \\
				maximum\_partial &  Abdelkader & Alexandre & Abdelkader & Abdelkader \\
				\hline
			\end{tabular}
		\caption{Récapitulatif des contributions au projet}
		\label{contributions}
\end{table}

Comme on peut le voir, les personnes qui se sont occupées de la rédaction d'un algorithme se sont aussi chargées de sa vérification et de sa complexité, tandis que le présent rapport est pour sa part l’œuvre d'une seule personne. \newline

\underline{Configuration de test :}
	\begin{itemize}[label=$\bullet$]
		\item{\emph{OS}} Windows 8.1 Professionnel 64 bits
		\item{\emph{CPU}} Intel Core i7-7400MQ 2.40GHz
		\item{\emph{GPU}} Nvidia GeForce GTX 760M
		\item{\emph{RAM}} 8GB DDR3
	\end{itemize}