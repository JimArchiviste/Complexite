\section{Alogirthmes de vérification}

Cette partie regroupe les deux algorithmes de vérification prenant en paramètre un graphe et un sous-graphe, et vérifiant si le sous-graphe est, respectivement, désert et maximal. 

Étant très similaires de fonctionnement, nous avons regroupé ces deux algorithmes dans la même partie. À noter qu'ils ne vérifient pas si le sous-graphe est bien un sous-graphe du graphe passé en paramètre : on suppose que les paramètres passés sont corrects.

\subsection{Propriété \og désert \fg}

\subsubsection{Principe}

\begin{verbatim}
booléen is_desert(graphe G, sous-graphe S){
    booléen desert = vrai
    entier n = taille de G
    matrice arc = ensemble des arcs de G
    pour i de 0 à n
        pour j de i+1 à n
            si (S[i]==1 et S[j]==1 et arc[i][j]==1) alors 
                desert = faux
                sortir des boucles
            fsi
        fpour
    fpour
    retourner desert
}
\end{verbatim}
Le principe de cet algorithme est de parcourir l'ensemble des arcs possibles à l'aide d'une double boucle et de vérifier, pour chaque arc :
\begin{enumerate}
	\item si ses deux extrémités sont présentes dans le sous-graphe
	\item s'il est présent dans le graphe
\end{enumerate}
Si tel est  le cas, cela veut dire que le sous-graphe n'est pas désert (deux de ses sommets sont reliés entre eux), on sort donc des boucles et on retourne faux.

Si rien ne nous arrête, alors on retourne vrai. En pratique pour éviter l'usage d'un \verb|break|, les boucles sont des \verb|while| qui vérifient à chaque itération la valeur du booléen.

\subsubsection{Validité}

La rigueur d'une preuve formelle n'est pas requise pour ce projet, or si l'on ne va  pas jusque là, il n'y a que peu à rajouter : l'algorithme applique exhaustivement la définition d'un sous-graphe désert. 

Tous les arcs  possibles sont dûment vérifiées compte tenu de cette définition (deux sommets d'un sous-graphe désert ne peuvent pas être reliés par un arc du graphe) ; le seul cas non-traité possible est celui où le sous-graphe comporte des sommets qui ne feraient pas partie du graphe de départ, dans cette éventualité ces somemts sont simplement ignorés. 

\subsubsection{Complexité}

La taille de G (le nombre de sommets qui le compose) notée n s'obtient en temps constant car elle fait partie de la structure représentant un graphe, de même que la matrice arc. 

Le pire des cas possibles est celui où le sous-graphe est bel et bien désert : il faudra parcourir l'intégralité des deux boucles pour arriver à cette conclusion, sachant que le traitement à l'intérieur des boucles est lui aussi en temps constant (trois tests). On a donc une complexité $C_1(n)$ de la forme suivante :

$3\sum\limits_{k=0}^{n-1} k = \frac{3n(n+1)}{2}$

Cela étant de l'ordre de $n^2$, on obtient la complexité suivante pour notre algorithme : 

$\theta(n^2)$

\emph{Temps d'exéc !!!} 

\subsection{Propriété \og maximal \fg}

\subsubsection{Principe}

\begin{verbatim}
booléen is_maximal(graphe G, sous-graphe S){
    booléen maximal = vrai
    entier n = taille de G
    matrice arc = ensemble des arcs de G
    si(is_desert(G,S) == 0) maximal = faux
    sinon
        entier i=0
        tant que (i<n et maximal==vrai)
            si(S[i] == 0)
                S[i] = 1
                maximal = is_desert(G,S)
                S[i] = 1
            fsi
        ftq
    fsi
    retourner maximal
}
\end{verbatim}
Le principe de cet algorithme est de parcourir l'ensemble des sommets de G non-présents dans S, et pour chacun d'entre eux effectuer la procédure suivante :
\begin{enumerate}
	\item ajouter ce sommet à S
	\item regarder si S est désert 
	\item retirer ce sommet de S
\end{enumerate}
Si une fois au moins on obtient un S désert dans l'étape 2, cela veut dire que notre sous-graphe n'est pas désert maximal et donc on quitte la boucle et renvoie faux, sinon on consomme les n tours de boucle puis on renvoie vrai. 

\subsubsection{Validité}

On sait qu'un sous-ensemble d'un sous-graphe désert sera lui-même désert. Ainsi pour un sous-graphe désert composé de k sommets il n'y a que 2 possibilités, soit il est maximal, soit il est strictement inclus dans un sous-graphe désert maximal M. 

En ajoutant chaque sommet de G un à un à S, on obtient au moins une fois M, soit un sous-graphe S' tel que $S \subset S' \subset M$.

De ce fait tester le non-respect de la propriété \og désert \fg{} pour chaque sous-graphe de taille k+1 contenant S nous permet de savoir si ce dernier est maximal ou pas.

\subsubsection{Complexité}

Le pire des cas est celui où S est désert maximal et ne comporte qu'un seul sommet (le cas de S vide impliquerait que G soit vide aussi, donc que n==0, ce qui fait que les boucles de la forme \verb|pour i de 0 à n| seraient somme toute assez rapides à exécuter). 

Nous ferons dans ce cas n tours de boucle, dont n-1 mènent à un appel à \verb|is_desert| dont la complexité est en $\theta(n^2)$. Il y a de  plus un appel à \verb|is_desert| en-dehors de la boucle ce qui nous donne la complexité suivante pour \verb|is_maximal| : 

$\theta(n^3)$

\emph{Temps d'exéc !!!} 